\documentclass[utf8,compress]{beamer}
\usepackage[francais]{babel}

% thème et couleur
\usetheme{Warsaw}
%~ \usecolortheme{wolverine}

% sommaire qu'avec les noms des parties
\setcounter{tocdepth}{1}

% sujet du diapo
\newcommand{\slidesubject}{Extension et révision évolutive d’un réseau}
\subtitle{\slidesubject}
\subject{\slidesubject}

% titre du diapo
\title{BE - Architecture de réseaux}

% auteurs
\author{Prénom \textsc{Nom}, Prénom \textsc{Nom}}

% formalités
\institute{
    Université Toulouse III - Paul Sabatier\\
    Master STRI\\
    \vspace{0.8em}
    \emph{À l'attention de} \\
    M.~Prénom \textsc{Nom} \\
    M.~Prénom \textsc{Nom}
}

% date
\date{\today}

%% Définition du répertoire contenant les images
\graphicspath{{images/}}

%% Numérotation des diapos :
\expandafter\def\expandafter\insertshorttitle\expandafter{%
    \insertshorttitle\hfill%
    \insertframenumber{}%
}


%~ ---------------------------------------------------------------------
% DEBUT DIAPORAMA
\begin{document}


%~ ---------------------------------------------------------------------
% TITRE
\begin{frame}
    \titlepage
\end{frame}


%~ ---------------------------------------------------------------------
% INTRODUCTION
\section{Introduction}

\begin{frame}{Introduction}
    \begin{block}{Mise en contexte}
        Lorem ipsum dolor sit amet, consectetur adipisicing elit, sed doeiusmod tempor incididunt ut labore et dolore magna aliqua.
        Ut enimad minim veniam, quis nostrud exercitation ullamco laboris nisi utaliquip ex ea commodo consequat.
    \end{block}
    \begin{block}{Quel sont les contraintes ?}
        Lorem ipsum dolor sit amet, consectetur adipisicing elit, sed doeiusmod tempor incididunt ut labore et dolore magna aliqua.
        Ut enimad minim veniam, quis nostrud exercitation ullamco laboris nisi utaliquip ex ea commodo consequat.
    \end{block}
\end{frame}


%~ ---------------------------------------------------------------------
% PLAN
\begin{frame}{Plan}
    \tableofcontents
\end{frame}


%~ ---------------------------------------------------------------------
%% Un rappel du plan sera affiché à chaque début de section.
\AtBeginSection[]
{
    \begin{frame}<beamer>
        \frametitle{Plan}
        \tableofcontents[currentsection]
    \end{frame}
}


%~ ---------------------------------------------------------------------
% PARTIE
\section{Schéma actuel}

%
\subsection{sous partie 1}
\begin{frame}[containsverbatim]{Sous Partie 1}
    \begin{block}{Bloc normal}
         C'est un joli bloc bleu.
    \end{block}
    \begin{exampleblock}{Bloc exemple}
         Le vert signifie que c'est un exemple.
    \end{exampleblock}
    \begin{alertblock}{Bloc alerte}
         Attention c'est rouge !
    \end{alertblock}
\end{frame}

%
\subsection{sous partie 2}
\begin{frame}{Sous Partie 2}
    \begin{figure}[h]
        \center
        \includegraphics[width=\textwidth]{image.png}
    \end{figure}
\end{frame}



%~ ---------------------------------------------------------------------
% PARTIE
\section{Objectifs d'évolution}

%
\subsection{sous partie 1}
\begin{frame}{Sous Partie 1}
    \begin{itemize}
        \item Une simple liste,
        \item avec deux éléments.
    \end{itemize}
    \begin{block}{Implication}
        \begin{itemize}
        \item cause
        \item[$\Rightarrow$] conséquence
        \end{itemize}
    \end{block}
\end{frame}

%
\subsection{sous partie 2}
\begin{frame}{Sous Partie 2}
    \begin{columns}
        \begin{column}{0.7\textwidth}
            Cette page est constituée de\dots
            \begin{itemize}
                \item deux colonnes ;
                \item un liste dans la première colonne ;
                \item une image dans la deuxième colonne ;
                \item un bloc.
            \end{itemize}
        \end{column}
        \begin{column}{0.3\textwidth}
            \begin{figure}[h]
                \includegraphics[width=3cm]{image.png}
            \end{figure}
        \end{column}
    \end{columns}

    \vspace{1em}

    \begin{block}{Le bloc}
        Ut eum praeceptum possemus fuit diligendo aliquando diligentiam ut fuit hoc quam inimicitiarum possemus praeceptum.
    \end{block}
\end{frame}




%~ ---------------------------------------------------------------------
% CONCLUSION
\section{Conclusion}
\begin{frame}{Conclusion}
    Bla bla bla, mr freeman... Bla bla bla...
\end{frame}


%~ ---------------------------------------------------------------------
% FIN DIAPORAMA
\end{document}
