\section{La première section}

%
\subsection{Une sous section}

On peut mettre des mots en \emph{italique},
en \textsc{petites Majuscules} ou
en \texttt{largeur fixe (machine à écrire)}.

Voici un deuxième paragraphe avec une formule mathématique simple : $e = mc^2$.

Un troisième avec des \og guillemet français \fg{}.


%
\subsubsection{Écrire en anglais}

\foreignlanguage{english}{Do you speak French? Does anybody here speak french?}


%
\subsection{Listes}

\begin{itemize}
\item Liste classique ;
\item un élément ;
\item et un autre élément.
\end{itemize}
\vspace{\parskip} % espace entre paragraphes

\begin{enumerate}
\item Une liste numéroté
\item deux
\item trois
\end{enumerate}
\vspace{\parskip}

\begin{description}
\item[Description] C'est bien pour des définitions.
\item[Deux] Ou pour faire un liste spéciale.
\end{description}
\vspace{\parskip}


%
\subsection{Références}

Voici une référence à l'image de la figure \ref{latex} page \pageref{latex} et une autre vers la partie \ref{p2} page \pageref{p2}.

On peut citer un livre\,\up{\cite{lpp}} et on précise les détails à la fin du rapport dans la partie références.


%
\subsection{Note de bas de page}

Voici une note\,\footnote{Texte de bas de page} de bas de page.
Une deuxième\,\footnotemark{} déclarée différemment.
La même note\,\footnotemark[\value{footnote}] que précédemment.

\footnotetext{Il a deux références vers cette note}


%
\subsection{Figure}

\begin{figure}[!ht]
    \center
    \includegraphics[]{./images/LaTeX_logo.png}
    \caption{latex | taille original}
    \label{latex}
\end{figure}

\begin{figure}[!ht]
    \center
    \includegraphics[width=0.5\textwidth]{./images/LaTeX_logo.png}
    \caption{latex | 50\% de la largeur de la page}
\end{figure}
