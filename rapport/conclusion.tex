\section*{Conclusion}
\addcontentsline{toc}{section}{Conclusion}

Notre bureau d'étude, nous propose quelques petits détails d'évolution.
En contrepartie, le manque d'informations provenant du cahier des charges annoncé, nous limite sur des recherches en terme de budget, de sécurité, et de topologie désirée.
Nous avons donc proposé une solution de qualité en restant raisonnable sur les coûts.

Bien qu'un hôpital ait besoin d'un réseau performant, la sécurité est aussi importante pour les patients et leur renseignements santé.
Ici on a privilégié le besoin métier pour faciliter les personnels et répondre aux besoins désirés.

Nous avons donc réunis les points importants pour répondre aux exigences en terme de sécurité (haute disponibilité), en terme de solutions économiques sous forme de tableaux, de schémas logiques et matériels, et aussi une justification des choix des équipements.


\section*{Perspectives d'évolution}
\addcontentsline{toc}{section}{Perspectives d'évolution}


En terme d’évolution il est bien possible qu’à l’avenir, le centre hospitalier désire modifier son architecture. Les débits deviennent de plus en plus conséquents, la sauvegarde par cloud se fait ressentir et la virtualisation reflète de plus en plus le mode de fonctionnement des entreprises pour limiter les coûts des équipements, gagner de l’espace, et ne plus avoir à les manager.

La problématique est que nous connaissions pas l’architecture présente dans l’ancien bâtiment. De ce fait nous pouvons parler que le PABX fonctionnel pourra être à l’avenir virtualisé ou bien il pourra être remplacé pour installer de la VOIP. La VOIP est en format numérique et limite les frais d’opérateur télécom.

La sortie pour accéder à Internet limite les utilisateurs à vouloir avoir plus d’échanges. Imaginons que le centre hospitalier s'agrandit ou qu’un nombre important d’utilisateur se connecte simultanément, l’accès serait bien moins accessible. Pour cela deux accès Internet serait préjudiciable pour répartir les requêtes.

L’espace de stockage pourrait lui aussi être limité. Les patients s’enchainent et leur profil s’enregistre dans la base de données. Au fur et à mesure malgré le rafraichissement des données, la base deviendra obsolète. Un serveur SAN ou bien un stockage cloud pourrait alléger l’organisation des ressources et des informations pour éviter tout dysfonctionnement futur ou pannes d’un tel réseau de stockage.

L’équilibrage de charge des routeurs dans l’ancien bâtiment est inconnu. Le fait de positionner une fibre optique entre les deux batiments, il serait nécessaire de savoir concrètement que le routeur actuel soit compatible en gigabit. En effet le nouveau bâtiment serait donc bridé au pire des cas.
La première solution est de remplacer ce routeur s’il ne convient pas au débit souhaité. Sinon la deuxième solution est de positionner un routeur temporaire afin d’installer un convertisseur à ce même routeur.

%
