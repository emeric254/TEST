\section*{Introduction} % Pas de numérotation
\addcontentsline{toc}{section}{Introduction} % Ajout dans la table des matières

%
%
Une expansion d'une clinique, dédiée aux maladies des voies respiratoires, voit le jour à 50 mètres d'un des bâtiments déjà existant.
Nous sommes chargés de réaliser l'étude d'architecture réseau à implanter dans ce nouveau bâtiment.

%
%
Ce réseau devra répondre à une certaine tolérance aux pannes puisque utilisé à des fins médicales.
Une interconnexion avec le bâtiment adjacent sera aussi nécessaire.
Dans l'architecture réseau actuelle le cœur de réseau et l'accès à Internet se trouvent dans le bâtiment adjacent.
Le déploiement de la nouvelle portion de réseau ne devra avoir aucune incidence sur le réseau déjà existant de la clinique.

%
%
Le nouveau bâtiment est composé de 6 étages ayant chacun différents usages.
Ces différences entre étages seront un point de départ important pour établir l'architecture réseau :
par exemple certains équipements médicaux nécessitent d'être inter-connectés, d'autres ne doivent en aucun cas être parasités pour assurer leur fonctionnement.

%
%
