\section*{Introduction} % Pas de numérotation
\addcontentsline{toc}{section}{Introduction} % Ajout dans la table des matières

%
%

Une expansion d'une clinique, dédiée aux maladies des voies respiratoires, voit le jour à 50 mètres d'un des bâtiments déjà existant.
Nous sommes chargés de réaliser l'étude d'architecture réseau à implanter dans ce nouveau bâtiment.

%
%

Ce réseau devra répondre à une certaine tolérance aux pannes puisque utilisé à des fins médicales et nécessitera une interconnexion avec le bâtiment adjacent.
Dans l'architecture réseau actuelle le cœur de réseau et l'accès à Internet se trouvent dans le bâtiment adjacent.
Le déploiement de la nouvelle portion de réseau ne devra avoir aucune incidence sur le réseau déjà existant de la clinique.

%
%

Sur le nouveau bâtiment il est nécessaire de définir et prendre en compte les contraintes imposées par les personnes, les matériels et leurs usages.
Chaque étage est dédié à des usages différents :
\begin{itemize}
\item Le niveau -2 contient le parking et les vestiaires du personnel;
\item Le niveau -1 contient le service de radiologie, avec des salles de scanners et des blocs opératoires;
\item Le rez-de-chaussée contient la salle d'accueil et le local technique principal;
\item Le niveau 1 contient les bureaux du personnel et les salles de réunion;
\item Les niveaux 2 à 4 contiennent les chambres des patients.
\end{itemize}
On s'aperçoit que ces différences entre étages seront un point de départ important pour l'architecture du réseau :
certains équipements médicaux nécessitent d'être inter-connectés, d'autres ne doivent en aucun cas être parasités pour assurer leur fonctionnement.

%
%
